\documentclass{article}
\usepackage{tikz}
\usetikzlibrary{shapes, positioning}

\begin{document}

$$Ax^2+By-C=0$$

Which can be rewritten to:

$$Ax^2+By = C$$

From this, we can construct a circuit:

\begin{tikzpicture}
    % First Layer
    \node (input1) at (3, 7) {$x$};
    \node (input2) at (5, 7) {$y$};
    \node (A) at (1, 7) {$A$};
    \node (B) at (7, 7) {$B$};
    
    % Second Layer
    \node[draw, rectangle] (mul21) at (3, 5.5) {$\times$};
    \node[draw, rectangle] (mul22) at (6, 5.5) {$\times$};

    \draw[->] (input1) -- (2, 7) |- (mul21);
    \draw[->] (input1) -- (4, 7) |- (mul21);

    \draw[->] (input2) -- (5, 6.5) |- (mul22);
    \draw[->] (B) -- (7, 6.5) |- (mul22);

    % Third Layer
    \node[draw, rectangle] (mul31) at (2, 4) {$\times$};

    \draw[->] (mul21) -- (3, 4) |- (mul31);
    \draw[->] (A) -- (1, 4) |- (mul31);

    % Fourth Layer
    \node[draw, rectangle] (add41) at (4, 2.5) {$+$};

    \draw[->] (mul31) -- (2, 3.5) |- (add41);
    \draw[->] (mul22) -- (6, 3.5) |- (add41);

    % Fifth Layer
    \node (output) at (4, 1) {$C$};

    \draw[->] (add41) -- (output);
\end{tikzpicture}

Choosing concrete values for $A = 3, B = 5, C = 47$:

\begin{tikzpicture}
    % First Layer
    \node (input1) at (3, 7) {$x$};
    \node (input2) at (5, 7) {$y$};
    \node (A) at (1, 7) {$3$};
    \node (B) at (7, 7) {$5$};
    
    % Second Layer
    \node[draw, rectangle] (mul21) at (3, 5.5) {$\times$};
    \node[draw, rectangle] (mul22) at (6, 5.5) {$\times$};

    \draw[->] (input1) -- (2, 7) |- (mul21);
    \draw[->] (input1) -- (4, 7) |- (mul21);

    \draw[->] (input2) -- (5, 6.5) |- (mul22);
    \draw[->] (B) -- (7, 6.5) |- (mul22);

    % Third Layer
    \node[draw, rectangle] (mul31) at (2, 4) {$\times$};

    \draw[->] (mul21) -- (3, 4) |- (mul31);
    \draw[->] (A) -- (1, 4) |- (mul31);

    % Fourth Layer
    \node[draw, rectangle] (add41) at (4, 2.5) {$+$};

    \draw[->] (mul31) -- (2, 3.5) |- (add41);
    \draw[->] (mul22) -- (6, 3.5) |- (add41);

    % Fifth Layer
    \node (output) at (4, 1) {$47$};

    \draw[->] (add41) -- (output);
\end{tikzpicture}

Now, I want to prove that I know an input to the circuit, $x$, s.t. the circuit is satisfied.

\begin{tikzpicture}
    % First Layer
    \node (input1) at (3, 7) {$x$};
    \node (input2) at (5, 7) {$y$};
    \node (A) at (1, 7) {$3$};
    \node (B) at (7, 7) {$5$};
    
    % Second Layer
    \node[draw, rectangle] (mul21) at (3, 5.5) {$\times$};
    \node[above left=0.01cm of mul21] {$a_1$};
    \node[above right=0.01cm of mul21] {$b_1$};
    \node[below right=0.01cm of mul21] {$c_1$};
    
    \node[draw, rectangle] (mul22) at (6, 5.5) {$\times$};
    \node[above left=0.01cm of mul22] {$a_2$};
    \node[above right=0.01cm of mul22] {$b_2$};
    \node[below right=0.01cm of mul22] {$c_2$};

    \draw[->] (input1) -- (2, 7) |- (mul21);
    \draw[->] (input1) -- (4, 7) |- (mul21);

    \draw[->] (input2) -- (5, 6.5) |- (mul22);
    \draw[->] (B) -- (7, 6.5) |- (mul22);

    % Third Layer
    \node[draw, rectangle] (mul31) at (2, 4) {$\times$};
    \node[above left=0.01cm of mul31] {$a_3$};
    \node[above right=0.01cm of mul31] {$b_3$};
    \node[below right=0.01cm of mul31] {$c_3$};

    \draw[->] (mul21) -- (3, 4) |- (mul31);
    \draw[->] (A) -- (1, 4) |- (mul31);

    % Fourth Layer
    \node[draw, rectangle] (add41) at (4, 2.5) {$+$};
    \node[above left=0.01cm of add41] {$a_3$};
    \node[above right=0.01cm of add41] {$b_3$};
    \node[below right=0.01cm of add41] {$c_3$};

    \draw[->] (mul31) -- (2, 3.5) |- (add41);
    \draw[->] (mul22) -- (6, 3.5) |- (add41);

    % Fifth Layer
    \node (output) at (4, 1) {$47$};

    \draw[->] (add41) -- (output);
\end{tikzpicture}

\end{document}
